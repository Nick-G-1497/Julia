
\begin{DoxyRefList}
\item[Class \mbox{\hyperlink{classEulersSpiral_1_1EulersSpiral}{Eulers\+Spiral.Eulers\+Spiral}} ]\label{todo__todo000001}%
\Hypertarget{todo__todo000001}%
transcibe this to the complex plains using identities 

find those identities  
\item[Namespace \mbox{\hyperlink{namespaceJulia}{Julia}} ]\label{todo__todo000002}%
\Hypertarget{todo__todo000002}%
Get 4K images rendering and have it have a time resolution of every second (360 \mbox{\hyperlink{namespaceJulia}{Julia}} sets per unit circle). This will take a lot of computation however we know that it is in fact finite and will take fewer than 183 trillion z squared plus c\textquotesingle{}s, however, in practice it is much lower since most of the unit circle compromises Fatou Dust and therefore diverges pretty quickly. 

dynamic color grading -\/ Currently the image rendering is handled by matplotlib pyplot. It is apparent the color grading happens in discrete boundaries as such so much of the resolution of the julia sets are lost via the color grading as much of the unit circle compromises Fatou Dust. As such a better implementation would be to implement color grading manually so that the colors fade more gradually and the resolution is maximized. 24 different color gradings need to be created -\/ one for each how of the day. 

imploment multi-\/threading so it builds multiple unit circle collections at the same time (can either uses mutexes have each rotation be its own independant resource) 

figure out why unit test succeed locally but fail when run with github actions 

collaborate with someone who has a strong background in algorithms to make it run faster 

eliminate the white borders surrounding the plot so it looks better as a background  
\item[Member \mbox{\hyperlink{classJulia_1_1Julia_a966f5090e8ab789ab45b8bbe84435da9}{Julia.Julia.color\+\_\+map\+\_\+\+P\+IL}} (self, itterations\+\_\+til\+\_\+divergence)]\label{todo__todo000003}%
\Hypertarget{todo__todo000003}%
fix spelling mistake  
\item[Namespace \mbox{\hyperlink{namespaceJuliasClock}{Julias\+Clock}} ]\label{todo__todo000004}%
\Hypertarget{todo__todo000004}%
Implement better file sturucture. Having all the folders for all of the color gradings in the root directory is kind of ugly. The program should be modified so that all the folders containing all of the pictures of all of julia sets are contained in like a backgrounds directory or something. Clean up the directory structure of the project. 

Consolidate all requirements into the requirements.\+txt file. Should be able to install all requirements seamlessly via pip install -\/r requirements.\+txt or the equivalent idea using conda. This will help make the code more productionalized. 

Trace out the unit circle once. Store a matrix of value which compromise the length of time which each point had to take to diverge. Map every other color grading onto it. Current P\+OC method of recalculating every julia set for every color grading is just shit.  
\item[Namespace \mbox{\hyperlink{namespaceWallpaperUpdater}{Wallpaper\+Updater}} ]\label{todo__todo000005}%
\Hypertarget{todo__todo000005}%
Make set\+\_\+wallpaper() work with any and all OS environments. Currently the wallpaper module only works with linux systems which are running a gnome environment, however, \mbox{\hyperlink{namespaceWallpaperUpdater}{Wallpaper\+Updater}} should be agnostic of the the environment of which is running and should still work. 
\end{DoxyRefList}